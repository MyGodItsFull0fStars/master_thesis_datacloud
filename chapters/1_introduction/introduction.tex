\chapter{Introduction}

    This is the introduction chapter

        \section{Motivation and Scope}
        \label{sec:motivation-and-scope-introduction}

        % this is a test \cite{datacloudAbout}
        \section{Research Problems}
        \label{sec:research-problems-introduction}

        \section{Research Objectives}
        \label{sec:research-objectives-introduction}

        \section{Research Methodology}
        \label{sec:research-methodology-introduction}
        
        \section{Thesis Outline}
        \label{sec:thesis-outline-introduction}

            First, in chapter \nameref{ch:background} the necessary background regarding monitoring, computation on large scale distributed systems, a scheduling and adaptation approach and forecast prediction with machine learning is explained, followed by related work in machine learning based forecast prediction and findings in publicly available data traces.
            In the chapter \nameref{ch:model-methodology} the methodology of this thesis is explained. 
            In chapter \nameref{ch:architecture-and-implementation} the architecture of the software is explained, followed by the preprocessing of data traces mentioned in \nameref{sec:public-cloud-provider-traces-in-available-data-related-work} and the adaptation approach used.
            Next, in chapter \nameref{ch:evaluation-and-results} an analysis on the available data is done, then the evaluation setup on how to set up the software is explained, followed by the different scenarios that were evaluated. 
            Finalizing the thesis with the chapter \nameref{ch:conclusions-and-future-work} that contains the conclusions about the findings of the evaluation in \nameref{sec:conclusions} as well as the \nameref{sec:future-work} that mentions possible improvements upon the current state of the software.


    % \bibliographystyle{ieeetr}
    % \bibliography{references}
