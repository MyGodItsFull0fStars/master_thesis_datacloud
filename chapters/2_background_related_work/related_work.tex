
    % --------------------------------------------------------------------
    % RELATED WORK

    % --------------------------------------------------------------------

    % Go based on the objectives or machine learning methods, research prediction....
    % Categorization of state of the art, how many papers are there for cloud, fog, or for machine learning stuff.

    % This section is for my competing approaches.



    \section{Public Cloud Provider Traces in Available Data}
    \label{sec:public-cloud-provider-traces-in-available-data-related-work}
    
    \section{Resource Prediction based on Machine Learning}
    \label{sec:resource-prediction-based-on-machine-learning-related-work}
    
    % what are the works that focus on the processing and tracing of resource utilisation
        Resource prediction based on machine learning involves the usage of forecasting machine learning algorithms that are capable of predicting future resource usage or the availability based on historical data.
        This can be applied to a variety of resources, such as energy, hardware utilisation.
        For example, in energy resource prediction, energy usage patterns may be used to train machine learning models that are capable of predicting energy consumption \cite{thonglekImprovingResourceUtilization2019} \cite{richDeepMindAIReduces2016}.

        There is a large variety of machine learning algorithms available that can be used for resource prediction, including linear regression, decision trees, random forests and neural networks.
        Neural networks that are especially promising for forecast prediction are \nameref{sec:rnn-background} and \nameref{sec:lstm-background}.
        
        The choice of algorithm highly depends on the specific problem and the type of data that is available.
        Overall, resource prediction based on machine learning can provide valuable insights and support informed decision making by allowing organizations to better plan and allocate resources.


    


            