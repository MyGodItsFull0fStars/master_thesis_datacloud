\chapter{State of the Art}
\label{ch:state-of-the-art}
    
    % what are the works that focus on the processing and tracing of resource utilisation
    
        Resource prediction based on machine learning involves the usage of forecasting machine learning algorithms that are capable of predicting future resource usage or availability based on historical data.
        This can be applied to a variety of resources, such as energy, and hardware utilisation.
        For example, in energy resource prediction, energy usage patterns may be used to train machine learning models that are capable of predicting energy consumption \cite{shapiEnergyConsumptionPrediction2021} \cite{richDeepMindAIReduces2016}.

        There is a large variety of machine learning algorithms available that can be used for resource prediction, including linear regression \cite{weisbergAppliedLinearRegression2005}, decision trees \cite{kotsiantisDecisionTreesRecent2013}, random forests \cite{breimanRandomForests2001} and neural networks \cite{andersonIntroductionNeuralNetworks1995}.
        Neural networks that are especially promising for forecast prediction are \nameref{sec:rnn-background} and \nameref{sec:lstm-background}.
        Improving resource utilisation is especially promising for large-scale computational systems such as data centres and cloud-based infrastructures. 
        The server load in a cloud computing environment is predicted in a hybrid Convolutional Neural Network - Long-Short Term Memory architecture in the paper \cite{patelHybridCNNLSTMModel2022}.
        % In \cite{thonglekImprovingResourceUtilization2019} a prediction model based on a long-short term memory machine learning is used that is capable of improving the resource utilisation compared to predictions based on user input.
        The choice of algorithm highly depends on the specific problem and the type of data that is available.
        Overall, resource prediction based on machine learning can provide valuable insights and support informed decision-making by allowing organizations to better plan and allocate resources.

        Tuli et al. \cite{tuliStartStragglerPrediction2021} proposed a novel prediction and mitigation method using an Encoder long-short-term memory (LSTM) model for large-scale cloud computing infrastructure. This method aims at reducing the application response time while maintaining the service level agreement between the application owner and resource provider.
        
        Thonglek et al. \cite{thonglekImprovingResourceUtilization2019} designed a neural network model based on LSTM as a type of Recurrent Neural Network (RNN) to predict resource allocation based on historical data. This model has two LSTM layers, each of which learns the relationship between i) allocation and usage, and ii) CPU and memory. It aims to improve resource utilization in data centres by predicting the required resource for each data pipeline.
        They analyse the CPU and memory allocation and utilisation separately and train the machine learning model to generate more accurate forecast predictions based on the provided sequential data and the allocated resource information.
        
        Oren et al. \cite{orenSOLOSearchOnline2021} represents resource utilisation as a combinatorial optimization problem and uses a set of Monte Carlo Decision Processes (MDP) \cite{jamesMonteCarloTheory1980}, Deep Q-learning and Graph Neural Networks as heuristics to improve the resource utilisation.

        Ngo et al. \cite{ngoContextualbanditAnomalyDetection2020} considered multiple anomaly detection deep neural network (DNN) models with varying complexity. Afterwards, the authors explored selecting one of the models to perform autonomous detection at most IoT, Edge, or Cloud layers. For the evaluations, the authors considered the devices such as NVIDIA Jetson-TX2 and NVIDIA Devbox with four GPU TitanX, respectively, as the Edge and Cloud server machines.

        Chen et al.~\cite{chenIRAFDeepReinforcement2019,chenIntelligentTaskOffloading2020} proposed a learning-based method that generates resource allocation decisions to minimize latency and power consumption. Intelligent resource allocation framework (iRAF) resource allocation action is predicted and obtained through self-supervised learning, where the training data is generated from the searching process of the Monte Carlo tree search (MCTS) algorithm.

        Tan and Hu~\cite{huMobilityawareEdgeCaching2018} used deep reinforcement learning to formulate the resource allocation optimization problem, where the parameters of caching, computing, and communication are optimized jointly.
