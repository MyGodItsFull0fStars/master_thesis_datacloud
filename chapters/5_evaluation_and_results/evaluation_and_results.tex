\chapter{Evaluation and Results}
\label{ch:evaluation-and-results}

\section{Trace Data}
\label{sec:data-analysis-evaluation}

  % \subsection{Structure of the Trace Data}
  This section contains the analysis of the Alibaba datasets that are used in section \ref{sec:evaluation-setup}. This data analysis builds on top of data analysis done in \cite{fengcunDeepJSJobScheduling2023}.

  \subsection{CPU Data Analysis}
  \label{sec:cpu-data-analysis}

    In table \ref{tab:actual-cpu-utilisation} the actual CPU utilisation of all tasks is shown.
    The dataset contains $5000$ elements and the CPU utilisation is represented in percentage, 
    thus a CPU utilisation of $200$ states that a task utilised two full CPU cores on average.

    \begin{table}
      \centering
      \caption{Actual CPU Utilisation}
      \label{tab:actual-cpu-utilisation}
      \begin{tabular}{|l|r|}
        \toprule
        {} &  CPU Utilisation (in \%) \\
        \midrule
        mean  &           364.386 \\
        std   &           559.695 \\
        min   &             0.000 \\
        25\%   &            89.336 \\
        50\%   &           165.370 \\
        75\%   &           455.047 \\
        max   &          7133.872 \\
        \bottomrule
      \end{tabular}
    \end{table}
    
    The mean of the CPU utilisation in this dataset is $364.386$, 
    so on average, a task required $559.695$ CPU cores in percentage for its execution.
    The standard deviation is approximately $5.6$ CPU cores per task, therefore the \emph{average} distance from the mean value is $5.6$ CPU core utilisation.
    In at least one case, no CPU cores were utilised, as can be seen in the minimum value of zero in the table.
    The first quartile shows that not even a full CPU core is used, and the second quartile that $1.6$ CPU cores are utilised.
    Next, the third quartile shows a CPU utilisation of $4.55$ CPU cores and finally, the maximum CPU utilisation used is approximately $71$ CPU cores. This high CPU utilisation as the maximum value is due to a task being deployed as many instances on multiple devices and the CPU utilisation being summed up for the task.
  

  \subsection{Memory Data Analysis}
  \label{sec:memory-data-analysis}

    In table \ref{tab:actual-memory-utilisation} the actual memory utilisation of all tasks is shown.
    This is the same dataset as is used for the CPU data analysis in section \ref{sec:cpu-data-analysis} and contains the corresponding memory utilisation of the tasks. Memory utilisation is represented in gigabytes (GB). 
    Therefore, a memory utilisation of $10$ means that a task required $10$ GB at maximum for finishing its execution.

    \begin{figure}
      \centering
      \caption{Actual Memory Utilisation}
      \label{tab:actual-memory-utilisation}
      \begin{tabular}{|l|r|}
        \toprule
        {} &  Memory Utilisation (in GB) \\
        \midrule
        mean  &             8.835 \\
        std   &            10.552 \\
        min   &             0.023 \\
        25\%   &             3.013 \\
        50\%   &             4.680 \\
        75\%   &            11.855 \\
        max   &           251.584 \\
        \bottomrule
      \end{tabular}
    \end{figure}
    The mean memory utilisation in this dataset is approximately $8.8$ GB for an average task.
    % TODO does this actually make sense from a statistical standpoint?
    The standard deviation is $10.552$, thus the distribution of memory utilisation compared to the mean is rather high if the quartiles are taken into account, especially the third quartile, which has a value of $11.855$ and only closely peaks the standard deviation.
    The minimum is shown as being about $23$ megabytes for a task in the dataset.
    The maximum memory utilisation is approximately $250$ gigabytes, and same as for the CPU utilisation, a task might consist of multiple instances, and their memory utilisation is summed up similarly.
    

  % \subsection{Alibaba Resource Analysis}
  % \label{sec:alibaba-resource-analysis-data-analysis}



% \section{Introduction}
\section{Evaluation Setup}
\label{sec:evaluation-setup}
% How everything was set up (kubernetes, ml,...)

  In this section, the evaluation setup will be described in detail. 
  This involves what tools and technologies were used and how they were combined in order to evaluate our test cases. Additionally, the \nameref{sec:forecasting-metrics-evaluation-setup} that are used are briefly described since they are crucial for the forecast prediction performance comparisons.
  
  \subsection{Hardware}
  \label{sec:hardware}

    The training of the \nameref{sec:lstm-background} model was done on a GPU server provided by ITEC, University of Klagenfurt. The server specifications are as follows:

    \begin{center}
      \begin{tabular}{| l | l |}
        \hline
        \textbf{Processor}   &   Intel Xeon Gold 5218 CPU @ 2.30GHz (64 cores) \\ \hline
        \textbf{GPU}         &   2 x NVIDIA Quadro RTX 8000 GPU (48 GB RAM)    \\ \hline
        \textbf{Main Memory} &   754 GB                                        \\ \hline
        \textbf{Operating System} &  Ubuntu Linux 18.04 LTS                    \\ \hline

        \hline
      \end{tabular}
      \end{center}




  \subsection{LSTM Model Setup}

  \subsection{LSTM Hyperparameters}
  \label{sec:lstm-hyperparameters-evaluation-setup}


  \subsection{Weights \& Biases}
  \label{sec:wandb-evaluation-setup}
    
    Weights \& Biases (W\&B)\footnote{https://wandb.ai} is a software company that provides an AI platform for machine learning and deep learning. Their platform provides tools for tracking experiments, visualizing models, and collaborating with team members.
    The platform provides a centralized repository for all of the information related to a machine learning project, including code, data, models, and results.
    W\&B offers a range of features to help users better understand their models, including interactive visualizations of model architecture, weight distributions, and training metrics. The platform also provides a suite of tools for tracking experiments, which makes it easy to compare different models and understand the impact of changes to the code or data.

    % The W\&B platform is designed to make it easier for data scientists, machine learning engineers, and researchers to manage the complexity of developing and deploying machine learning models. 


    % Overall, W\&B provides a comprehensive solution for machine learning and deep learning development, and can be a valuable tool for organizations looking to streamline their machine learning operations and improve their ability to develop high-quality models.
  
  \subsection{Forecasting Metrics}
  \label{sec:forecasting-metrics-evaluation-setup}

    Forecasting metrics \cite{botchkarevPerformanceMetricsError2018} are measures used to evaluate the accuracy of forecasting models. These metrics are used to compare different models, assess the quality of the forecasts, and identify areas for improvement.

    % Maybe add those metrics? 


    % Theil's U-Statistic: measures the accuracy of a forecast relative to the accuracy of a naive forecast.

    % Mean Absolute Scaled Error (MASE): measures the accuracy of a forecast relative to the accuracy of a naïve forecast.

    The choice of forecasting metric will depend on the specific goals and requirements of the forecasting task. Some metrics may be more appropriate for certain types of data or models, while others may be better suited for comparing different models.
    Overall, the use of appropriate forecasting metrics is critical for evaluating the accuracy of forecasting models and for identifying areas for improvement.

    \subsubsection{Mean Absolute Percentage Error (MAPE)}
    \label{sec:mape-metrics-evaluation}
      Mean Absolute Percentage Error (MAPE) \cite{demyttenaereMeanAbsolutePercentage2016} is a commonly used metric for evaluating the accuracy of forecasting models. It measures the average percentage difference between the predicted values and the actual values.
      The formula for MAPE is:

      \begin{pabox}{Mean Absolute Percentage Error}
        $$MAPE = \frac{1}{n} \times \sum \left|\frac{Actual - Predicted}{Actual}\right| \times 100$$
      \end{pabox}
      where $n$ denotes the number of data points and $Actual$ and $Predicted$ are the actual and predicted values, respectively.
      MAPE provides a percentage error, which makes it easy to interpret and compare the accuracy of forecasting models. 
      
      However, there are some limitations to using MAPE, such as the fact that it can become undefined when the actual value is zero (in our case if there is no resource utilisation at some time step $i$), and it can be sensitive to outliers.
      Despite these limitations, MAPE is a widely used metric for evaluating forecasting models, especially if it is important to understand the relative magnitude of the errors in the predictions. 
      Overall, MAPE provides a useful way to measure the accuracy of forecasting models and can help to identify areas for improvement in the model or the data.


    \subsubsection{Symmetric Mean Absolute Percentage Error (sMAPE)}
    \label{sec:smape-metrics-evaluation}
    
      Symmetric Mean Absolute Percentage Error (sMAPE) \cite{kreinovichHowEstimateForecasting2014} is a metric used for evaluating the accuracy of forecasting models. It measures the average percentage difference between the predicted values and the actual values and is symmetrical in that it treats positive and negative errors equally.
      In our evaluation of the prediction performance, this is especially useful since both over and under-utilisation are present in all prediction variants.
      The formula for SMAPE is:

      \begin{pabox}{Symmetric Mean Absolute Percentage Error}
        $$sMAPE = \frac{1}{n} \times \sum \left|\frac{Actual - Predicted}{\left(Actual + Predicted\right) \div 2}\right| \times 100$$
      \end{pabox}
      where $n$ denotes the number of data points and $Actual$ and $Predicted$ are the actual and predicted values, respectively.
      sMAPE provides a percentage error similar to \nameref{sec:mape-metrics-evaluation}, which makes it easy to interpret and compare the accuracy of forecasting models. Unlike MAPE, sMAPE is symmetrical, since it doesn't become undefined when the actual value is zero, and another difference is that it is less sensitive to outliers.
      Overall, sMAPE is a useful metric for evaluating the accuracy of forecasting models and can provide valuable information for understanding the relative magnitude of the errors in the predictions.


    % Mean Absolute Error (MAE): measures the average magnitude of the error between the predicted values and the actual values.

    \subsubsection{Mean Square Error (MSE)}
    \label{sec:mse-metric-evaluation}

      Mean squared error (MSE) was used as the loss function for all trained models except for the model in evaluation scenario \ref{sec:training-with-custom-loss-function-evaluation-scenarios}.
      As already stated in section \ref{sec:penalty-mse-loss-function-architecture-and-implementation}, the MSE measures the average of the squares of the errors compared to ground truths (the actual values), which is the average squared difference between the actual value $y$ and the estimated value $\hat{y}$.

      \begin{pabox}{Mean Squared Error}
      \label{def:mean-squared-error-evaluation}
        $$MSE = \frac{1}{N} \sum_{i = 1}^{N}\left(y_i - \hat{y}_i\right)^2$$
      \end{pabox}
      As this metric is also used in the training loop of each LSTM model and also is widely used, it is also included in the metrics of each evaluation scenario.
    

    \subsubsection{Root Mean Square Error (RMSE)}
    \label{sec:rmse-metrics-evaluation}

      Root Mean Square Error (RMSE) \cite{chaiRootMeanSquare2014} is a metric used for evaluating the accuracy of forecasting models. It measures the average magnitude of the error between the predicted values and the actual values and provides a useful way to compare the magnitude of the errors in different models.
      The formula for RMSE is:
      \begin{pabox}{Root Mean Square Error}
        $$RMSE = \sqrt{\frac{\sum_{i = 1}^{N}\left(Predicted_i - Actual_i\right)^2}{N}}$$
      \end{pabox}
      where $n$ denotes the number of data points and $Actual$ and $Predicted$ are the actual and predicted values, respectively.
      RMSE provides a measure of the magnitude of the errors in the predictions, with lower values indicating a more accurate model. 
      RMSE is particularly useful when the goal is to minimize the magnitude of the errors in the predictions.
      One important thing to note about RMSE is that it is sensitive to the scale of the data. This means that it is more appropriate to use RMSE when the scale of the actual and predicted values is similar.
      Because of the scale sensitivity of RMSE, the number of actual and predicted values is equivalent for all our evaluation scenarios in order to be able to use RMSE for a comparison between different approaches.
      Overall, RMSE is a widely used and useful metric for evaluating the accuracy of forecasting models and can provide valuable information for understanding the magnitude of the errors in the predictions.

% \subsubsection{Theil's U-Statistic}

% Theil's U-Statistic is a measure of the accuracy of a forecasting model relative to a naïve forecast. The naïve forecast is a simple forecast that uses the average or last observed value as the prediction for all future time points. Theil's U-Statistic is used to compare the accuracy of a forecast with the accuracy of the naïve forecast.

% The formula for Theil's U-Statistic is:

% U = (RMSE of the forecast) / (RMSE of the naïve forecast)

% where RMSE stands for Root Mean Squared Error.

% Theil's U-Statistic is a unitless measure, with values ranging from 0 to infinity. A value of 0 indicates that the forecast is no better than the naïve forecast, while a value greater than 1 indicates that the forecast is more accurate than the naïve forecast.

% Theil's U-Statistic is particularly useful when the goal is to compare the accuracy of a forecast with the accuracy of a simple, baseline forecast. It provides a useful way to evaluate the improvement in accuracy achieved by using a more sophisticated forecasting model.

% Overall, Theil's U-Statistic is a valuable tool for evaluating the accuracy of forecasting models, and for comparing the accuracy of different models relative to a simple, baseline forecast.
    \subsubsection{Resource Wastage}
    \label{sec:resource-wastage-metric-evaluation}

      The resource wastage for a single task is defined as the difference between the predicted and the actual value. Since wastage refers to utilising more resources than necessary, only predicted values that are higher than the actual values are considered for this calculation.
      % Calculated by subtracting the surface area of allocated - predicted.

\section{Evaluation Scenarios}
\label{sec:evaluation-scenarios}

% the different evaluations I did like task type or batch size
% what is common in the scenarios (or some of them)
    In evaluation scenarios 1-4, the forecast performance of different machine learning training scenarios is evaluated and compared with each other. Each of them builds on top of the previous evaluation scenario to compare the improvement.

  \subsection*{0. User-Defined Hardware Utilisation}
  \label{sec:user-defined-hardware-utilisation}

      This evaluation scenario analysis the user-defined hardware utilisation. 
      In this variant, users are required to estimate the hardware utilisation of their tasks when providing those tasks to the GPU clusters.
      This user-defined prediction is the comparison base for improvement in the evaluation scenarios that will follow.

      \subsubsection{CPU Prediction Analysis}
      \label{sec:user-defined-cpu-prediction-analysis}

      \begin{table}
        \centering

        \begin{tabular}{|l|rr|}
          \toprule
          {} &  User Predicted CPU &  Actual CPU Usage \\
          \midrule
          mean       &         569.86 &            364.39 \\
          std        &         300.24 &            559.70 \\
          min        &           5.00 &              0.00 \\
          25\%        &         400.00 &             89.34 \\
          50\%        &         600.00 &            165.37 \\
          75\%        &         600.00 &            455.05 \\
          max        &        3200.00 &           7133.87 \\
          OA percentage &         78.60 &             \\
          UA percentage &          21.40 &             \\
          \bottomrule
          \end{tabular}
      \end{table}

      \begin{tabular}{lrrrr}
        \toprule
        {} &        MSE &     RMSE &     MAPE &   SMAPE \\
        \midrule
        User CPU &  356923.33 &  597.431 &  944.986 &  99.682 \\
        \bottomrule
        \end{tabular}

    \subsubsection{Memory Prediction Analysis}
    \label{sec:user-defined-memory-prediction-analysis}



  \subsection*{1. Simple Feature Set}
  \label{sec:simple-feature-set-evaluation-scenarios}
    
    The simple feature set denotes a \nameref{sec:lstm-background} model that was trained with a feature set that only contains the allocated CPU and memory a user provided for the submitted task as well as the capacity of the worker pool resource the task should be deployed to. Additionally, since LSTMs are designed to handle sequential data, the order of the incoming tasks is also a piece of important information for the model, and thus cannot be randomized as is common for other machine learning algorithms.

    \begin{tabular}{|lrr|}
    \toprule
    {} &  Actual CPU Usage &  Predicted CPU Usage \\
    \midrule
    mean  &        218.992136 &           597.048348 \\
    std   &        397.247882 &           774.300155 \\
    min   &          1.022901 &            58.950184 \\
    25\%   &         54.799999 &            97.586464 \\
    50\%   &        103.868813 &           243.593903 \\
    75\%   &        189.713745 &           627.078918 \\
    max   &       5068.459473 &          5793.996094 \\
    \bottomrule
    \end{tabular}

  \subsection*{2. Adding Task Knowledge}
  \label{sec:adding-task-knowledge-evaluation-scenarios}

    In this evaluation scenario, the feature set used to train the LSTM model contains additionally to the feature set arguments of the \nameref{sec:simple-feature-set-evaluation-scenarios} also \emph{task knowledge}. This task knowledge refers to the type a task is classified as. The task knowledge was first extracted from the data set and transformed with the \nameref{sec:one-hot-encoding-preprocessing-architecture} method. 
    The tasks are mapped to a finite set of a one-hot encoded feature vector and their corresponding index is represented as the value $1$.
    The additional information about the task type enables the LSTM model to better discern the actual utilisation that is required for each task.
    The task knowledge also provides information regarding the order of tasks, since patterns of reoccurring tasks also help the model to generate more accurate predictions in theory.

    \paragraph{CPU Prediction Evaluation}
    \label{par:cpu-prediction-evaluation-task-knowledge}

      Providing categorical data is able to positively influence the prediction performance of the machine learning model. This is also seen in the predictions of our LSTM model. The additional knowledge about what type of tasks are present in the pipeline enabled the ML model to create more accurate predictions of the CPU utilisation for each task.

      As can be seen in the figure 
      % \ref{fig:cpu-comparision-no-tasks-vs-tasks} 
      the CPU utilization could be predicted more accurately if the model was trained with the additional information about which type of task will be allocated onto the system.
      Compared to the ML model trained without this additional knowledge (further on called NT-model, short for No-Tasks-model), the model trained with task knowledge is able to predict the required allocation on a finer granularity.
      This is observed in the predictions as more changes in the prediction values occur compared to the NT-model, which often seems to predict an average value over a greater time span rather than predicting the allocation for each task itself.
      The WT-model is also more sensitive to great utilization spikes and periods of utilization close to zero compared to the NT-model.

    \paragraph{Memory Prediction Evaluation}
    \label{par:memory-prediction-evaluation-task-knowledge}

      Similar to the prediction of CPU utilization, memory utilization was improved if the ML model was trained with additional information about the task type. This enabled the model to achieve a more accurate prediction of the actual memory usage compared to the NT-model.

      This improvement is even more apparent compared to the allocated memory data, which heavily over-allocates the required amount of memory. Compared to the NT-model and allocated memory the WT-models predictions are able to reduce the gap between predicted memory utilization and actual memory utilization the furthest.

  \subsection*{3. Adding Instance Knowledge}
  \label{sec:adding-instance-knowledge-evaluation-scenarios}

    In this evaluation scenario, the feature set used to train the LSTM model contains additionally to the feature set arguments of the \nameref{sec:adding-task-knowledge-evaluation-scenarios} also information about the number of instances that are required to be deployed. While the knowledge about the number of instances is similarity one-hot encoded as is the task knowledge (see \ref{sec:adding-task-knowledge-evaluation-scenarios}), adding one feature vector column for every instance number would have resulted in exploding the feature data set with 800 additional columns since the minimum instance number is $1$ and the maximum is $800$.
    For this reason, the feature vector for the one-hot encoded instances was first clustered into $10$ buckets in $50$ iterations by k-means clustering \cite{hartiganAlgorithm136Kmeans1979}.

  \subsection*{4. Training with Custom Loss Function}
  \label{sec:training-with-custom-loss-function-evaluation-scenarios}

      As discussed in the previous evaluation scenarios, 
      additional information regarding a task (see sections \ref{sec:adding-task-knowledge-evaluation-scenarios} and \ref{sec:adding-instance-knowledge-evaluation-scenarios}) will result in closer predictions to the actual value but also the LSTM model is more likely to predict values that are smaller than the actual value.
      % TODO explain why this is a bad thing
      
      Therefore, a custom loss function was designed and implemented (see \nameref{sec:penalty-mse-loss-function-architecture-and-implementation}) to counter the tendency of the LSTM model with an increasing number of feature columns to predict values lower than the actual value.

  \subsection*{5. Training Time}
  \label{sec:training-time-evaluation-scenarios}

    The evaluation scenario regarding the training time measures the time that is required for the LSTM model to train for one epoch. The number of data points is fixed and the same for all evaluation scenarios and LSTM models, so ...
    % TODO explain that only the difference between the models is calculated
    % plus I take the average of the training time per epoch

  \subsection*{6. Inference Time}
  \label{sec:inference-time-evaluation-scenarios}

    The evaluation scenario regarding the inference time measures the time that is required for the LSTM model to forward a prediction.
    The time measurement is done on different batch sizes. This is important to be able to forward the resource utilisation prediction to the scheduler in an acceptable time frame.
    A low inference time by the LSTM model is mandatory in order for its predictions to be usable by the scheduler of the \nameref{sec:scheduling-and-adaptation-background} approach. Too high inference times can lead to task pipeline congestion if the scheduler has to wait for tasks to arrive from the LSTM model.
    The required inference time depends on three major factors, the batch size, i.e., the number of tasks provided at once to the LSTM model to calculate a prediction of their hardware utilisation.
    The next factor is the complexity of the LSTM model, where a more complex neural network architecture also requires more internal calculations while forwarding the input data (the tasks of the pipeline). The complexity of the LSTM model can be derived from the number of layers and LSTM cells as well as used hyper-parameters, most importantly the \emph{hidden size} of the LSTM layers, which has the most impact both on the internal complexity and the amount of space the LSTM model requires.
    Finally, the hardware the LSTM model resides on has a major influence on the time that is required to infer a prediction from input data. Deep learning nowadays heavily relies on the usage of modern GPUs that use so-called \emph{tensor cores} specifically build for machine learning tasks to accelerate both the training and the inference times.


  \subsection*{7. Prediction Performance Depending on Batch Size}
  \label{sec:prediction-performance-depending-on-batch-size-evaluation-scenarios}



    % Different batch sizes
    % Different models?



  The batch size not only influences the required training time and how much sampling data at once the model weights are updated on. The used batch size even has an impact on the prediction performance of the LSTM model.
    
  As can be seen in figure 
  % \ref{fig:comparison-of-500-1000-batch-size}, 
  
  the batch size used while training the LSTM model has a significant impact on the overall prediction performance.
  In the figure, the training batches have the sizes 500 (orange dashed line) and 1000 (blue dashed line) elements per batch, the actual CPU utilization is shown as the black line and the CPU cores (in \%) allocated by the user are shown in the green line. While the prediction with a batch size of 500 did overestimate the CPU utilization, it has some great prediction results in the largest utilization spike, where it did come closest to the actual CPU utilization and also corrected the CPU allocation in multiple instances to either correctly use less or more CPU cores for a task.
  
  % The training was done on a dataset size of 4000 elements total, which is only a small fraction of the actual dataset size.
  
  
  % \begin{figure}[h!]
  %     \centering
  %      \includegraphics[width=\columnwidth]{figures/500_1000bs_comparison.png}
  %     \caption{CPU Comparison of Batch Sizes 500 and 1000 - Part 1}
  %     \label{fig:comparison-of-500-1000-batch-size}
  % \end{figure}


  In figure 
  % \ref{fig:comparison-of-500-1000-batch-size-2},
   it is shown that the model with a batch size of 1000 is more accurately predicting lower utilization values and its overall bias in the dataset is to predict lower utilization values.
  The model trained with batch size 500 predicts a higher CPU utilization in general, but also did correctly increase/decrease the CPU utilization if the actual utilization was high/low for a longer period of time.

  The opposite is observed for the model with a batch size of 1000. 
  This model is more accurate in predicting short utilization spikes or declines compared to the model with a batch size of 500.
  
  % \begin{figure}[h!]
  %     \centering
  %      \includegraphics[width=\columnwidth]{figures/500_1000bs_comparison2.png}
  %     \caption{CPU Comparison of Batch Sizes 500 and 1000 - Part 2}
  %     \label{fig:comparison-of-500-1000-batch-size-2}
  % \end{figure}


  In figure 
  % \ref{fig:comparison-of-500-1000-batch-size-mem}, 
  the prediction of the memory utilization is shown.
  Opposed to the CPU utilization prediction, in the model trained with the smaller batch size is biased to predict lower values in general, and the model trained with a batch size of 1000 predicted higher utilization values. Also, it was able to produce accurate predictions of the memory utilization of tasks.
  

  % TODO find out if I should keep this section
  
  % \subsection*{8. Machine-Sorted vs. Time-Stamp-Sorted Data}
  % \label{sec:machine-sorted-vs-time-stamp-sorted-data-evaluation-scenarios}

  %   All of the above evaluation scenarios were done with time-stamp sorted data. Such datasets simply refer to an ordering of the data points that were sorted by the earliest arrival time first, followed by the machine those tasks were deployed to.
  %   In the machine-sorted dataset the priority of sorting is reversed, meaning that the data points were first sorted by each machine and then all elements were sorted by the earliest arrival time for each machine. 
  %   This was done to see if there are any improvements in the prediction performance if the data set is sorted by the machine parameter first and only then sorted by the time-stamp.

    
% \section{Monitoring}



% \section{Adaptation}
