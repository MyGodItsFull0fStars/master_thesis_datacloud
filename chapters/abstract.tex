\section*{Abstract}
  \label{sec:abstract}

This thesis tackles resource utilisation issues in large-scale distributed infrastructures comprising cloud, fog, and edge systems. Accurate resource utilisation estimates are essential for proper infrastructure functioning. However, current estimates tend to be overestimated by over 30\%, resulting in the allocation of more resources than necessary and leading to instability and inefficiency. Additionally, scheduling tasks with communication delays is a well-known NP-hard problem, both for homogeneous and heterogeneous resources. Machine learning approaches have been proposed, but there is no consensus on the best approach. Deep learning algorithms have shown promise in providing accurate predictors. This thesis proposes using Long-Short-Term Memory to train a resource utilisation estimator on large datasets and fine-tune it for specific infrastructures to reduce over-provisioning, increase resource utilisation, stabilize the infrastructure, and optimise task scheduling. The thesis focuses on providing a proof-of-concept solution, evaluating it on a simulated infrastructure, and discussing potential improvements and limitations. 


\section*{Keywords}
\label{sec:keywords}
Long-Short-Term Memory, machine learning, resource utilisation, cloud, fog, edge,  task scheduling, resource utilisation estimation.
